\documentclass[11pt]{article}

%%%%%%%%%%%%%%%%%%%%%%%%%%%%%%%%%%%%%%%%
% Packages
%%%%%%%%%%%%%%%%%%%%%%%%%%%%%%%%%%%%%%%%

\usepackage{oke-header-math}

%%%%%%%%%%%%%%%%%%%%%%%%%%%%%%%%%%%%%%%%
% Mathematics
%%%%%%%%%%%%%%%%%%%%%%%%%%%%%%%%%%%%%%%%



%%%%%%%%%%%%%%%%%%%%%%%%%%%%%%%%%%%%%%%%
% Title
%%%%%%%%%%%%%%%%%%%%%%%%%%%%%%%%%%%%%%%%

\title{Guide to training RBM/Dynamic Boltzmann Dist.}
\author{OKE}

%%%%%%%%%%%%%%%%%%%%%%%%%%%%%%%%%%%%%%%%
% Begin document
%%%%%%%%%%%%%%%%%%%%%%%%%%%%%%%%%%%%%%%%

\begin{document}

\maketitle

Key:
\begin{enumerate}
\item \textbf{Softmax/multi-valued}
\item \textbf{Layer-wise pre-training}: is very important. Train each RBM as a single layer, then propagate the data upward. Finally, train the whole network.
\begin{enumerate}
\item \textbf{Self-intersecting trajectories in the pre-training}:
\end{enumerate}
\item \textbf{Initial values for weights/biases}: A good value for the initial visible bias is $\log p_i/(1-p_i)$ where $p_i$ is the fraction of training data examples where unit $i$ is on. A good value for the initial weight is close to $0$. A good value for the initial hidden bias is $0$. A very negative value (e.g. $-4$) can be used here as well to enforce a kind of sparsity constraint, but this can make the weight values very sensitive.
\item \textbf{Nesterov acceleration}: start with a low value $0.5$; later $0.9$ can be used to improve training.
\end{enumerate}


\end{document}